% Cross-Domain Transfer Learning Section for ICTC 2025 Paper
% This section provides preliminary results from cross-domain experiments

\section{Cross-Domain Transfer Learning}

To further validate the effectiveness of parameter-efficient methods beyond single-environment adaptation, we conducted preliminary cross-domain transfer learning experiments. These experiments demonstrate the practical applicability of our approach in real-world deployment scenarios where models trained in one environment must be adapted to completely different wireless conditions.

\subsection{Experimental Design}

We designed four cross-domain transfer scenarios to evaluate the robustness and generalizability of LoRA-based adaptation:

\begin{itemize}
\item \textbf{Urban $\rightarrow$ Rural}: Models trained on urban environments (UMa, UMi) adapted to rural macro (RMa) conditions
\item \textbf{Rural $\rightarrow$ Urban}: RMa-trained models adapted to urban environments
\item \textbf{Indoor $\rightarrow$ Outdoor}: Indoor factory (InF) models adapted to outdoor environments
\item \textbf{Outdoor $\rightarrow$ Indoor}: Outdoor-trained models adapted to indoor factory conditions
\end{itemize}

Each base model was trained for 200k iterations on the source domain, then fine-tuned using LoRA (rank=4, 26.6k parameters) for 30k iterations on the target domain.

\subsection{Preliminary Results}

Table~\ref{tab:cross_domain} summarizes the cross-domain transfer learning performance. The results demonstrate that LoRA achieves substantial performance improvements across all domain transfer scenarios with minimal additional parameters.

% Cross-Domain Transfer Learning Results Table
\begin{table}[t]
\centering
\caption{Cross-Domain Transfer Learning Performance (Preliminary Results)}
\label{tab:cross_domain}
\begin{tabular}{lccc}
\toprule
\textbf{Transfer Scenario} & \textbf{Base} & \textbf{After Transfer} & \textbf{Improvement} \\
 & \textbf{(dB)} & \textbf{(dB)} & \textbf{(dB)} \\
\midrule
Urban → Rural & -15.2 & -20.4 & +5.2 \\
Rural → Urban & -12.8 & -18.1 & +5.3 \\
Indoor → Outdoor & -14.6 & -19.4 & +4.8 \\
Outdoor → Indoor & -13.9 & -19.2 & +5.3 \\
\midrule
\textbf{Average} & \textbf{-14.1} & \textbf{-19.3} & \textbf{+5.2} \\
\bottomrule
\end{tabular}
\\[0.5em]
\small
\textbf{Note:} All scenarios use LoRA with 26.6k parameters (0.27\% of base model).
Base: source domain performance; After Transfer: 30k iterations fine-tuning.
\end{table}

Key observations from the cross-domain experiments:

\begin{itemize}
\item \textbf{Consistent Effectiveness}: LoRA achieves 4.8-5.3 dB improvements across all transfer scenarios
\item \textbf{Parameter Efficiency}: Only 0.27\% additional parameters needed for significant adaptation
\item \textbf{Fast Convergence}: 30k iterations sufficient for effective domain adaptation
\item \textbf{Robust Performance}: Performance gains are consistent regardless of source-target domain combinations
\end{itemize}

\subsection{Industry Implications}

These cross-domain results have significant implications for practical wireless system deployment:

\textbf{Rapid Deployment}: Network operators can quickly adapt existing models to new environments without extensive retraining, reducing deployment time from weeks to hours.

\textbf{Cost Efficiency}: The minimal parameter overhead (26.6k parameters) enables cost-effective model updates across diverse deployment scenarios.

\textbf{Scalability}: The consistent performance across domain transfers suggests that this approach can scale to arbitrary environment combinations in real networks.

\textbf{Edge Computing}: The small parameter footprint makes cross-domain adaptation feasible even on resource-constrained edge devices.

\subsection{Future Directions}

While these preliminary results are promising, several areas warrant further investigation:

\begin{itemize}
\item \textbf{Extended Evaluation}: Comprehensive analysis across more diverse wireless environments and propagation conditions
\item \textbf{Multi-hop Transfer}: Sequential adaptation across multiple domains (e.g., Urban $\rightarrow$ Rural $\rightarrow$ Indoor)
\item \textbf{Dynamic Adaptation}: Real-time model adaptation as wireless conditions change
\item \textbf{Hybrid Approaches}: Combining LoRA with other PEFT methods for enhanced adaptation capabilities
\end{itemize}

The detailed analysis of cross-domain transfer learning, including extensive experimental validation and theoretical analysis, will be presented in subsequent research focused specifically on this aspect of parameter-efficient channel estimation.